\documentclass[a4paper,french]{article}


%Tableaux
\usepackage{array}

%\usepackage{pifont}
\usepackage[french]{babel}
\usepackage{times}
\usepackage{babel}
\usepackage[utf8]{inputenc}
\usepackage[T1]{fontenc}
\usepackage{amsfonts}
\usepackage{amssymb}

% Algorithmes
\usepackage[linesnumbered, french]{algorithm2e}

% Lien hypertexte
\usepackage{hyperref}

%\setbeamercolor{titre}{bg=red,fg=white}
%\setbeamercolor{texte}{bg=red!10,fg=black}
%\beamerboxesdeclarecolorscheme{blocbleu}{blue}{yellow}
%\definecolor{vertmoyen}{RGB}{51,110,23}
%\beamerboxesdeclarecolorscheme{blocvert}{vertmoyen}{white}

%Package Tikz
\usepackage{tikz}

%\usetheme{Warsaw}

%\hypersetup{pdfpagemode=FullScreen}


\colorlet{titre}{red}

\definecolor{rouge}{HTML}{880000}
%\setbeamercolor{background canvas}{bg=white}
\newcounter{exonum}
\newcommand{\Exo}{\addtocounter{exonum}{1}{{\large \textbf {Exercice \theexonum .}}}}
\newcommand{\Titre}[1]{\textbf{{\large{#1}}}}

% Formatage de la page
% --------------------
\pagestyle{headings}
\oddsidemargin = 0.2in % 0in
\evensidemargin = 0in
\textwidth = 6.0in % 6.3in 348pt %
\topmargin = -3cm
\headheight = 0.2in
\headsep = 0.5in
\textheight = 270 mm % 9.4in 


\begin{document}
\pagestyle{empty}
\parindent 0mm

\begin{center}
\Titre{1$^{\grave{e}re}$ NSI  -  Algorithmes de tri}
\\[5mm]
\end{center}
\hrule
\vspace*{5mm}

%\section{Exercice 1}
\Exo

Par groupe de 4 élèves, effectuer les actions suivantes :
\begin{enumerate}
\item Dans un jeu de 32 cartes, regrouper les cartes d'une même couleur (cœur, carreau, trèfle, pique).
\item Récupérer chacun une couleur (à garder jusqu'à la fin de la séance).

\item Mélanger chaque paquet.

\item Trier dans l'ordre croissant (l'as est la carte de plus grande valeur).

\item Exposer et échanger entre vous la manière de faire. Utiliser le cadre réponse ci-dessous pour en rendre compte.

% Cadre réponse
\noindent\fbox{\hbox to 15 cm{\vbox to 5cm{}}}
\medskip

\item Mélanger à nouveau les cartes et recommencer le tri en commençant par placer à gauche la carte de plus petite valeur. Est-ce la méthode utilisée précédemment ?

% Cadre réponse
\noindent\fbox{\hbox to 15 cm{\vbox to 2cm{}}}
\medskip

\item Rechercher sur Internet les méthodes de tri par insertion et de tri par sélection. Exposer ci-dessous le principe de ces méthodes. Est-ce que ces méthodes correspondent à celle(s) que vous avez expérimentée(s) aux questions (4) et (6) ?

% Cadre réponse
\noindent\fbox{\hbox to 15 cm{\vbox to 7cm{}}}
\end{enumerate}
\bigskip

$\hookrightarrow$ POINT COURS : METHODES DE TRIS PAR INSERTION ET PAR SÉLECTION (diaporama - 10 min)
\pagebreak

%\section{Exercice 2}
\Exo

À l'aide des manipulations et des recherches faites dans l'exercice précédent, compléter les algorithmes ci-dessous :

\begin{enumerate}
%\subsection{Tri sélection}
\item Algorithme du tri par sélection : 

\begin{algorithm}[H]
\SetKwInput{Entree}{Entrée}
\SetKwInput{Donnees}{Variables locales}
\DontPrintSemicolon
TriSelection ($S$ : Tab)\;
\Entree{\;\Indp
$S$ : tableau non trié d'entiers\;
}
\Sortie{\;\Indp
$S$ : tableau trié
}
\BlankLine
\DontPrintSemicolon
\Donnees{\;\Indp
$i$ : entier - compteur pour boucle\;
$j$ : entier - compteur pour boucle\;
$indice$ : entier - indice de l'élément le plus petit\;
}
\PrintSemicolon
\BlankLine
\Deb{
\BlankLine
\BlankLine
\BlankLine
\BlankLine
\BlankLine
\BlankLine
\BlankLine
\BlankLine
\BlankLine
\BlankLine
\BlankLine
\BlankLine
\BlankLine
\BlankLine
\BlankLine
\BlankLine
\BlankLine
\BlankLine
\BlankLine
\BlankLine
\BlankLine
\BlankLine
\BlankLine
\BlankLine
\BlankLine
\BlankLine
\BlankLine
\BlankLine
\BlankLine
\BlankLine
\BlankLine
\BlankLine
\BlankLine
\BlankLine
\BlankLine
\BlankLine
\BlankLine
\BlankLine
\BlankLine
\BlankLine
\BlankLine
\BlankLine
\BlankLine
\BlankLine
\BlankLine
\BlankLine
\BlankLine
\BlankLine
\BlankLine
\BlankLine
\BlankLine
\BlankLine
\BlankLine
\BlankLine
\BlankLine
\BlankLine
\BlankLine
\BlankLine
\BlankLine
\BlankLine
\BlankLine
\BlankLine
\BlankLine
\BlankLine
\BlankLine
\BlankLine
\BlankLine
\BlankLine
\BlankLine
\BlankLine
\BlankLine
\BlankLine
\BlankLine
\BlankLine
\BlankLine
\BlankLine
\BlankLine
\BlankLine
\BlankLine
\BlankLine
\BlankLine
\BlankLine
\BlankLine
\BlankLine
\BlankLine
\BlankLine
\BlankLine
\BlankLine
\BlankLine
}
\end{algorithm}

%\subsection{Tri insertion}
\item Algorithme du tri par insertion : 

\begin{algorithm}[H]
\SetKwInput{Entree}{Entrée}
\SetKwInput{Donnees}{Variables locales}
\DontPrintSemicolon
TriInsertion ($S$ : Tab)\;
\Entree{\;\Indp
$S$ : tableau non trié d'entiers\;
}
\Sortie{\;\Indp
$S$ : tableau trié
}
\BlankLine
\DontPrintSemicolon
\Donnees{\;\Indp
$i$ : entier - compteur pour boucle\;
$j$ : entier - compteur pour boucle\;
$valeur$ : entier - valeur de l'élément à déplacer par insertion\;
$indice$ : entier - indice futur de l'élément à déplacer par insertion\;
}
\PrintSemicolon
\BlankLine 
\Deb{
\BlankLine
\BlankLine
\BlankLine
\BlankLine
\BlankLine
\BlankLine
\BlankLine
\BlankLine
\BlankLine
\BlankLine
\BlankLine
\BlankLine
\BlankLine
\BlankLine
\BlankLine
\BlankLine
\BlankLine
\BlankLine
\BlankLine
\BlankLine
\BlankLine
\BlankLine
\BlankLine
\BlankLine
\BlankLine
\BlankLine
\BlankLine
\BlankLine
\BlankLine
\BlankLine
\BlankLine
\BlankLine
\BlankLine
\BlankLine
\BlankLine
\BlankLine
\BlankLine
\BlankLine
\BlankLine
\BlankLine
\BlankLine
\BlankLine
\BlankLine
\BlankLine
\BlankLine
\BlankLine
\BlankLine
\BlankLine
\BlankLine
\BlankLine
\BlankLine
\BlankLine
\BlankLine
\BlankLine
\BlankLine
\BlankLine
\BlankLine
\BlankLine
\BlankLine
\BlankLine
\BlankLine
\BlankLine
\BlankLine
\BlankLine
\BlankLine
\BlankLine
\BlankLine
\BlankLine
\BlankLine
\BlankLine
\BlankLine
\BlankLine
\BlankLine
\BlankLine
\BlankLine
\BlankLine
\BlankLine
\BlankLine
\BlankLine
\BlankLine
\BlankLine
\BlankLine
\BlankLine
\BlankLine
\BlankLine
\BlankLine
\BlankLine
\BlankLine
\BlankLine
}
\end{algorithm}
\medskip

$\hookrightarrow$ POINT COURS : CORRECTION (diaporama - 10 min)
%\medskip
\pagebreak

%\subsection{Dérouler à la main}
\item Dérouler \og à la main \fg les algorithmes précédents avec les deux combinaisons de cartes suivantes :
% Choix d'une combinaison de 4 cartes et de 8 cartes

\begin{itemize}
\item[$\triangleright$] \framebox{\texttt{10}} \hspace{2mm} \framebox{\texttt{8}} 
\hspace{2mm} \framebox{\texttt{D}}  \hspace{2mm} \framebox{\texttt{9}}
\medskip

\item[$\triangleright$] \framebox{\texttt{V}} \hspace{2mm} \framebox{\texttt{AS}} 
\hspace{2mm} \framebox{\texttt{8}}  \hspace{2mm} \framebox{\texttt{D}} \hspace{2mm} \framebox{\texttt{10}} \hspace{2mm} \framebox{\texttt{7}} 
\hspace{2mm} \framebox{\texttt{9}}  \hspace{2mm} \framebox{\texttt{R}}
\end{itemize}
\medskip
 
\begin{enumerate}
\item Pour chaque combinaison, compter le nombre de comparaisons et de déplacements effectués puis compléter le tableau ci-dessous.  Un algorithme paraît-il plus performant que l'autre ?

\setlength{\extrarowheight}{2mm}
\begin{tabular}{|c|c|c|c|c|}
\cline{2-5}
\multicolumn{1}{c|}{} &\multicolumn{2}{|c|}{Tri sélection} & \multicolumn{2}{|c|}{Tri insertion}
\\ \cline{2-5}
\multicolumn{1}{c|}{}  & Nb comparaisons & Nb déplacements & Nb comparaisons & Nb déplacements \\
\hline
4 cartes &  &   &   &   \\[2mm] \hline
8 cartes &   &   &   &   \\[2mm] \hline
\end{tabular}
\\[2mm]
\item Les algorithmes ont-ils triés correctement les combinaisons aléatoires de cartes ? Est-ce une preuve totale ?
\end{enumerate}
\medskip

\item Recommencer les questions (a) et (b) avec une autre combinaison de 8 cartes non triées. 

\setlength{\extrarowheight}{2mm}
\begin{tabular}{|c|c|c|c|c|}
\cline{2-5}
\multicolumn{1}{c|}{} &\multicolumn{2}{|c|}{Tri sélection} & \multicolumn{2}{|c|}{Tri insertion}
\\ \cline{2-5}
\multicolumn{1}{c|}{}  & Nb comparaisons & Nb déplacements & Nb comparaisons & Nb déplacements \\
\hline
8 cartes &   &   &   &   \\[2mm] \hline
\end{tabular}
\\[2mm]

\item Que se passe-t-il lorsque la combinaison choisie est déjà triée ?

\setlength{\extrarowheight}{2mm}
\begin{tabular}{|c|c|c|c|c|}
\cline{2-5}
\multicolumn{1}{c|}{} &\multicolumn{2}{|c|}{Tri sélection} & \multicolumn{2}{|c|}{Tri insertion}
\\ \cline{2-5}
\multicolumn{1}{c|}{}  & Nb comparaisons & Nb déplacements & Nb comparaisons & Nb déplacements \\
\hline
8 cartes triées &   &   &   &   \\[2mm] \hline
\end{tabular}
\\[2mm]

\item D'autres exemples avec animation :
\begin{itemize}
\item \url{http://lwh.free.fr/pages/algo/tri/tri$\_$selection.html}
\item \url{http://lwh.free.fr/pages/algo/tri/tri$\_$insertion.html}
\end{itemize}
\end{enumerate}
\bigskip

$\hookrightarrow$ POINT COURS : PREUVE ET COMPLEXITÉ (diaporama)






\end{document}
